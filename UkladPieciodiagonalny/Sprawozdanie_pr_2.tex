\documentclass[11pt]{article}

\usepackage[T1]{fontenc}
\usepackage[polish]{babel}
\usepackage[utf8]{inputenc}
\usepackage{lmodern}
\selectlanguage{polish}

\usepackage{Sweave}
\begin{document}
\input{Sprawozdanie_pr_2-concordance}
\title{Metody Numeryczne \\ sprawozdanie projekt nr 2 }
\author{Bartosz Jamroży, grupa laboratoryjna nr 1}
\maketitle


\section*{Polecenie}
\begin{center}
\fbox{\includegraphics{polecenie2.png}}
\end{center}


\newpage
\section*{Program obliczeniowy}
Zaimplementowana funkcja obliczająca to 
\newline [x,det\_A,det\_A\_inverse] = RozwiazanieUkladuPieciodiagonalna(A,b)
\subsection*{Parametry}
  \subsubsection*{Wejście}
    \begin{itemize}
    \item A Macierz ukladu linowego \\ 
    Macierz kwadratowa pięciodiagonalna o elementach rzeczywistych 
    \item b Wyrazy wolne układu\\
      wektor liczb rzeczywistych
    \end{itemize}
  
  \subsubsection*{Wyjście}
    \begin{itemize}
    \item x Rozwiązanie układu \\ 
    wektor rzeczywisty 
    \item det \\
      wyznacznik macierzy A
    \item det\_inverse \\
      wyznacznik macierzy odwrotnej do A  
    \end{itemize}

  \subsection*{Opis działania}
    Funkcja RozwiazanieUkladuPieciodiagonalna składa się z następujących podfuncji:\\
    \indent PienciodiagonalnaJakoProstokatna(A) Zwraca macierz, której wiersze są           diagonalami 
    (odpowiednio rozszerzonymi zerami do długości n rozmiaru A, 
    diagonala 1 i 2 przedłużone z tyłu, 4 i 5 z przodu) z A
    Niezerowe elemnty poza diagonalami zostają zignorowane \\
    \indent EliminacjaGaussa(B,b) Dokonuje operacji na kwadratowym odpowiedniku A           odpowiadającej podstawowej eliminacji gausa
    na oryginalniej macierzy pięciodiagonalnej. \\
    \indent Obliczaniex(Be,be) Wyzancza rozwiazanie ukladu  

\newpage
\section*{Przykłady obliczeniowe}

\subsection*{Normalne działanie, krok po kroku}
Przykładowa losowa macierz pięciodiagonalna:
\begin{center}
\fbox{\includegraphics{A5.png}}
\end{center}

Oraz wektor wyrazów wolnych:
\begin{center}
\fbox{\includegraphics{b_.png}}
\end{center}

Po przekształceniu do postaci 5xn:
\begin{center}
\fbox{\includegraphics{B.png}}
\end{center}

Po elimancji gaussa:
\begin{center}
\fbox{\includegraphics{Be.png}}
\end{center}

Wyarazy wolne:
\begin{center}
\fbox{\includegraphics{be_.png}}
\end{center}

Teraz zgausowana macierz 5xn jest odpowiednikiem takiej kwadratowej macierzy(tylko dla zobrazowia co sie stało w obliczeniach,program nadal korzysta z formatu 5xn):
\begin{center}
\fbox{\includegraphics{Be_5xn.png}}
\end{center}

Ostateczny wektor rozwiązań:
\begin{center}
\fbox{\includegraphics[height=5cm,width=2cm]{x.png}}
\end{center}



\subsection*{Zero na diagonali}
Jako ze funcja realizowana jest poprzez podstawą eliminacje Gaussa nie jest odporna na pojawienie sie zera na diagonali.
W takim przypadku funcja zwróci wektor NaN-óW.

Macierz powstała z wyboru diagonali z macierzy pascala:
\begin{center}
\fbox{\includegraphics{P5.png}}
\end{center}

Jakiś wektor wyrazów wolnych (bez znaczenia jaki nie on jest tu problemem):
\begin{center}
\fbox{\includegraphics{p.png}}
\end{center}

Wynik:
\begin{center}
\fbox{\includegraphics[height=5cm,width=2cm]{xp.png}}
\end{center}



\subsection*{Macierz nie pięciodiagonalna}
Gdy jako maciez podana zostanie macierz nie pięciagonalna, alementy poza diagonalami zostaną zignorowane:
\begin{center}
\fbox{\includegraphics{A_A5_b.png}}
\end{center}

Wywołanie dla obu macierzy da ten sam efekt:\\
RozwiazanieUkladuPieciodiagonalna(A,b)==RozwiazanieUkladuPieciodiagonalna(A5,b)
\begin{center}
\fbox{\includegraphics[height=3cm,width=2cm]{xequalx5.png}}
\end{center}

\subsection*{Niepoprawne argumenty}
\begin{center}
\includegraphics{err1.png}
\end{center}
\begin{center}
\includegraphics{err2.png}
\end{center}

\section*{Analiza działania programu}
Wizualizacja błędu rozwiązywania układu dla loosowych macierzy różnych wymiarach. Błąd został wyliczony jako różnica napisanej funkcji do wbudowanej A\ b. Rozbierzności sa niewielki.
\begin{center}
\includegraphics{blad.png}
\end{center}
Po powiększeniu wykresu widać wartosci liczbowe
\begin{center}
\includegraphics{blad_p.png}
\end{center}


\end{document}
